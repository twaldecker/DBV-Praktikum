\documentclass[a4paper,11pt]{scrartcl}
\parindent 0pt
\usepackage[utf8]{inputenc}
\usepackage{amsmath}
\usepackage{amssymb}

%opening
\title{Digitale Bildverarbeitung - Lösung Blatt 3 (Fourier-Transf., Splines, DFT)}
\author{Thomas Waldecker\\Stefan Giggenbach}

\begin{document}

\maketitle

\newpage

\section{Aufgabe 1: Fourier-Beziehungen}
\subsection{1a) Verschiebung im Ortsbereich}

$\mathcal{F}\{f(x-\alpha,y-\beta)\}=\iint_{-\infty}^{\infty}f(x-\alpha,y-\beta)\cdot e^{-j2\pi(xu+yv)} dx dy$ \\

mit den Substitutionen $\xi=x-\alpha$ ($x=\xi+\alpha$) und $\eta=y-\beta$ ($y=\eta+\beta$) ergibt sich \\

$=\iint_{-\infty}^{\infty}f(\xi,\eta)\cdot e^{-j2\pi((\xi+\alpha)u+(\eta+\beta)v)} d\xi d\eta$ \\

durch Ausklammern der konstanten Faktoren erhält man \\

$=e^{-j2\pi(\alpha u+\beta v)}\iint_{-\infty}^{\infty}f(\xi,\eta)\cdot e^{-j2\pi(\xi u+\eta v)} d\xi d\eta$ \\

da das Integral der Fouriertransformation $\mathcal{F}\{f(\xi,\eta)\}=F(u,v)$ entspricht folgt \\

$\mathcal{F}\{f(x-\alpha,y-\beta)\}=F(u,v)\cdot e^{-j2\pi(\alpha u+\beta v)}$

\subsection{1b) Faltungssatz}

$\mathcal{F}\{h(x,y)\ast f(x,y)\}=\iint_{-\infty}^{\infty}[\iint_{-\infty}^{\infty}h(\chi,\psi)\cdot f(x-\chi,y-\psi) d\chi d\psi]\cdot e^{-j2\pi(xu+yv)} dx dy$ \\

durch Ausklammern und Umstellen der konstanten Faktoren erhält man \\

$=\iint_{-\infty}^{\infty}h(\chi,\psi)\cdot[\iint_{-\infty}^{\infty}f(x-\chi,y-\psi)\cdot e^{-j2\pi(xu+yv)} dx dy] d\chi d\psi$ \\

da das innere Integral mit der Beziehung aus Teilaufgabe 1a) der Fouriertransformation $\mathcal{F}\{f(x-\chi,y-\psi)\}=F(u,v)\cdot e^{-j2\pi(\chi u+\psi v)}$ entspricht folgt \\

$=\iint_{-\infty}^{\infty}h(\chi,\psi)\cdot F(u,v)\cdot e^{-j2\pi(\chi u+\psi v)} d\chi d\psi$ \\

durch Ausklammern des konstanten Faktor erhält man \\

$=F(u,v)\iint_{-\infty}^{\infty}h(\chi,\psi)\cdot e^{-j2\pi(\chi u+\psi v)} d\chi d\psi$ \\

da das Integral der Fouriertransformation $\mathcal{F}\{h(\chi,\psi)\}=H(u,v)$ entspricht folgt \\

$\mathcal{F}\{h(x,y)\ast f(x,y)\}=F(u,v)\cdot H(u,v)$

\newpage

\section{Aufgabe 2: Approximation von sinc(x) durch kubischen Spline}

\newpage

\section{Aufgabe 3: DFT/Abtastung/Abtasttheorem}

\end{document}
